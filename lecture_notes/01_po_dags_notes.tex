\documentclass{tufte-handout}
\usepackage{amsthm}
\usepackage{graphicx}
\usepackage{amsmath}
\usepackage{amssymb}
\usepackage{hyperref}
\usepackage{epigraph}

\hypersetup{
  colorlinks,
  urlcolor = blue,
  pdfauthor={Paul Goldsmith-Pinkham}
  pdfkeywords={econometrics}
  pdftitle={Lecture Notes for Applied Empirical Methods}
  pdfpagemode=UseNone
}
\newtheorem{ruleN}{Rule}
\newtheorem{thmN}{Theorem}
\newcommand{\bY}{\mathbf{Y}}
\newcommand{\bX}{\mathbf{X}}
\newcommand{\E}{\mathbb{E}}

\def\inprobHIGH{\,{\buildrel p \over \rightarrow}\,} 
\def\inprob{\,{\inprobHIGH}\,} 
\def\indistHIGH{\,{\buildrel d \over \rightarrow}\,} 
\def\indist{\,{\indistHIGH}\,}

\title{Lecture 1 - Potential Outcomes and Directed Acylic Graphs}
\author{Paul Goldsmith-Pinkham}
\date{\today}


\begin{document}

\maketitle



Not every economics research paper is estimating a causal quantity. But, the implication or takeaway of papers is (almost) always a causal one.  Causality lies at the heart of every exercise. \sidenote{``We do not have knowledge of a thing until we have grasped its why, that is to say, its cause.'' -- Aristotle}



 Goal for today's class:
  Enumerate tools used to discuss causal questions
Emphasize a \emph{multimodal} approach
Set terminology/definitions for future discussions

    \newthought{In his later books},
\cite{lewbel2019identification}

\begin{itemize}
  \item The true underpinnings of causality are nearly philosphical in nature
    \begin{itemize}
    \item If Aristotle didn't settle the question, neither will
      researchers in the 21t century
    \end{itemize}
  \item I will avoid many of the discussions, but my biases will show up in one or two settings
  \item Key point: economics research is messy, and a careful
    discussion of causality entails two dimensions:
    \begin{enumerate}
    \item A good framework to articulate your assumptions
    \item Readers that understand the framework
    \end{enumerate}
  \end{itemize}


 \newthought{Medical example} 
  \begin{itemize}
    \item  Two variables:
    \begin{itemize}
    \item $Y \in \{0, 1\}$: whether a person will get Covid-19
    \item $D \in \{0, 1\}$: whether a person gets a vaccine
    \end{itemize}
  \item Our question: does $D$ causally affect $Y$?
  \item \emph{Ignore the question of data for now} -- this is purely a
    question of what is knowable.
  \item ``The fundamental problem of causal inference'' (Holland 1986)
    is that for a given individual, we can only observe one world --
    either they get the vaccine, or they do not
  \end{itemize}
\bibliographystyle{plain}
\bibliography{lecture_note_bib}
\end{document}